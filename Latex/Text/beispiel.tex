\chapter{Dies ist ein Beispieldokument}
Dieses Dokument soll einen \texttt{kurzen} Überblick über das Arbeiten
mit \LaTeX \index{\LaTeX} und der Nutzung des verfügbaren Frameworks geben.

Dies ist ein Text \glossary{Text} mit Verweisen auf den Glossar.

\nomenclature{$Text$}{Definition}

Grundlegend gilt, dass wir stets um die Weiterentwicklung und
Aktualisierung dieses Frameworks bemüht sind. Sollten Ihnen Fehler
oder andere Missstände auffallen, so möchten wir Sie bitten uns diese
zu melden, damit wir die Möglichkeit zur Nachbesserung haben. Für eine
Benachrichtigung per Email an \url{felde@nm.ifi.lmu.de} sind wir
dankbar!

Der Nachfolgende Teil gliedert sich in zwei Teile: Die grundlegenden
Anforderungen zur erfolgreichen Nutzung unseres bereitgestellten
Rahmenwerks und nachfolgend eine ganz grobe Einleitung zur Nutzung von
\LaTeX \index{\LaTeX}.


\section{Anforderungen zur Nutzung des Rahmenwerks}

\subsection{Präambel}
Alle schriftlichen Arbeiten am Lehrstuhl Kranzlmüller sind mit \LaTeX
\index{\LaTeX} zu erstellen. Dies hat den Vorteil, dass die Dokumentation auf
vielen Betriebssystemplattformen einfach reproduziert und weiterverarbeitet
werden kann. Um ein einheitliches Layout zu gewährleisten, ist weiter dieser
zur Verfügung gestellte Style in nicht abgeänderter Form zu verwenden.

Das vorliegende Archiv enthält neben den LaTeX-Styles auch noch ein
Rahmenwerk, welches die Arbeit unter Linux vereinfachen soll. Das
gesamte Framework wird stetig weiterentwickelt und
verbessert. Deswegen sind wir über jeden per Email gemeldeten Fehler
und Missstand an \url{felde@nm.ifi.lmu.de} dankbar.


\subsection{Zur Nutzung}
Dieses Paket ist für die Nutzung unter Linux konzipiert. Es basiert
auf \texttt{pdflatex} und enthält einige hilfreiche Funktionen, die
zum Teil in Kapitel \ref{sec:nutzung} kurz umrissen werden.

Die Nutzung in anderen Umgebungen ist prinzipiell möglich, wird von
uns aber nicht weiter unterstützt. Bei der Erstellung von Arbeiten in
anderen Laufzeitumgebungen ist jedoch darauf zu achten, dass die hier
gemachten Layout-Vorgaben strikt eingehalten werden.

\subsubsection{Verzeichnisstruktur des Archivs}
Das vorliegende Archiv hat eine bestimmte Verzeichnisstruktur, die
unbedingt eingehalten werden muss, da sämtliche Arbeiten am Lehrstuhl
Kranzlmüller durch ein Publikationsverwaltungssystem veröffentlicht
werden. Dieses System erfordert unbedingt die vorgegebende
Struktur. Weitere Informationen hierzu finden sich auf unseren
Internetseiten.

\subsubsection{Kompilieren des Dokuments}
Für das Kompilieren der Ausarbeitung stehen maßgeblich 2 Makefiles zur
Verfügung. Durch ein Aufruf von \texttt{make} im Verzeichnis
.../Dokumentation/Latex/ wird das Dokument ausschließlich
kompiliert. Soll zusätzlich zum kompilieren des Dokuments auch die
Verteilung der erzeugten Dokumente auf die richtigen Verzeichnisse
unter Beachtung der korrekten Dateinamengebung geschehen, so muss
\texttt{make} im Verzeichnis .../Dokumentation/ aufgerufen
werden. Hierbei ist zu beachten, dass in der Datei
\texttt{Makefile.DEF} ein gültiger Schlüssel für die Variable \glqq
MASTER\grqq\ eingetragen worden ist. Einen eindeutigen Schlüssel
erhalten Sie bei Ihrem jeweiligen Betreuer der Arbeit.


\section{Nutzung von \LaTeX\ in Verbindung mit dem Rahmenwerk} 
\label{sec:nutzung}
Nachfolgend wird nur ein kleiner Überblick über ganz grundlegende
Möglichkeiten von \LaTeX \index{\LaTeX} gegeben. Für weitergehende
Informationen sei an dieser Stelle auf Kapitel \ref{subsec:howto}
verwiesen.


\subsection{Gliederung von Dokumenten}
Um ein Kapitel zu erzeugen wird der Befehl \textbackslash
chapter\{...\} verwendet. Unterkapitel können mit \textbackslash
section\{...\}, \textbackslash subsection\{...\} usw. eingefügt
werden.

Ein Absatz wird durch eine Leerzeile im Quelltext erreicht.

Fußnoten\footnote{Dies ist eine Fußnote.} können mit dem Befehl
\textbackslash footnote\{...\} generiert werden.


\subsection{Einfügen von Bildern}
\includeParpic{l}{width=0.2\textwidth}{demo_xosview}{demoParPic}{Um\-flos\-sene
PNG-Grafik}

\includeParPic{r}{.3\textwidth}{width=0.2\textwidth}{demo_xosview}{demoParPic_neu}{Um\-flos\-sene
PNG-Grafik}

Um eine Grafik wie Abbildung \ref{fig:demoGif} in das Dokument einzufügen, wird
der Befehl \textbackslash includeFigure \{\} \{\} \{\} \{\} \{\} verwendet. Die
Parameter sind dabei von links nach rechts betrachtet: \glqq
Positionsangabe\grqq, \glqq Größenangabe\grqq, \glqq Dateiname\grqq\ (ohne
Erweiter\-ung), \glqq label\grqq und \glqq Caption\grqq (eine kurze Beschreibung
der Grafik). Alle Bilddateien sind im Verzeichnis
.../Dokumentation/Latex/Bilder/ abzulegen. Bevorzugt sollten Vektorgrafiken im
PDF-Format eingepflegt werden. Es werden jedoch auch andere Bildformate wie
z.B. \texttt{*.fig}, \texttt{*.dia}, \texttt{*.svg}, \texttt{*.png},
\texttt{*.gif}, \texttt{*.jpg}, \texttt{*.tif}, \texttt{*.pdf} und
\texttt{*.eps} unterstützt. Abbildung \ref{fig:demoGif} zeigt ein Beispiel für
das Einbinden einer PNG-Grafik.

Vom Text umflossene Bilder können mit dem Kommando \textbackslash
includeParpic \{\} \{\} \{\} \{\} \{\} eingebettet werden. Die
Parameter sind dabei mit oben genannten identisch. Abbildung
\ref{fig:demoParPic} zeigt ein Beispiel dafür.
Als Alternative steht außerdem der Befehl \textbackslash includeParPic
\{\} \{\} \{\} \{\} \{\} \{\}  (zwei ,,P'') zur Verfügung.  Der zusätzliche Parameter
gegenüber \textbackslash includeParpic dient dazu Platz (in der Breite) für das Bild zu reservieren.
Abbildung~\ref{fig:demoParPic_neu} zeigt ein Beispiel, in dem ein größerer Abstand zwischen  Bildrand und Fließtext gewählt wurde als in Abbildung~\ref{fig:demoParPic}.

%Außerdem können hier
%mehrere Attribute statt der einfachen Bildgröße angegeben werden.  Die
%Parameter sind demnach: \glqq Positionsangabe\grqq, \glqq
%Breitenreservierung\grqq, \glqq Bildattribute\glqq, \glqq label\grqq, \glqq
%Dateiname\grqq und \glqq Caption\grqq. Abbildung \ref{fig:demoParPic} zeigt ein
%Beispiel dafür.

\includeFigure{t}{width=0.15\textwidth}{demo_clock}{demoGif}{Eine
  einfache PixelGrafik als PNG}

\subsection{Einfügen von Sourcecode-Beispielen}
In der Datei .../Dokumentation/Latex/main.tex muss man für jede
verwendeten Quellcodetyp die entsprechenden Definition für die
Syntaxhervorhebung geladen werden. Folgende Kommandos werden in diesem
Beispiel verwendet:

\setupVerbatim{tex}
\begin{verbatim}
\lstloadlanguages{C}
\lstloadlanguages{java}
\end{verbatim}

Die zur Verfügung stehenden Definitionen sind aus \cite{hein99} ersichtlich.

\subsubsection{Quellcode direkt im Text}
Um Quellcode direkt im Text verwenden zu können gibt es eine
spezielle Umgebung, die hier benutzt worden ist:

\setupVerbatimCaption{C}{Hello World in C}{hw-c}
\begin{verbatim}
#include<stdio.h>

int main( int argc, char* argv[])
{
        printf( "Hello world!\n");
}
\end{verbatim}

Die verwendeten Kommandos lauten:

\setupVerbatim{tex}
\begin{verbatim}
\setupVerbatimCaption{syntaxdef}{caption}{label}
\begin {verbatim}
    sourcecode
\end {verbatim}
\end{verbatim}

\subsubsection{Quellcode aus einem File}
Das folgende Programm wurde direkt aus einem File gelesen:

\inputListingCaption{helloWorld.java}{java}{Hello World in Java}{hw-java}

Das verwendete Kommando lautet:

\setupVerbatim{tex}
\begin{verbatim}
\inputListingCaption{filename}{syntaxdef}{caption}{label}
\end{verbatim}

Die eingelesene Datei liegt im Verzeichnis
.../Dokumentation/Latex/Code, welches automatisch durchsucht wird. 

Auch die Einbindung ohne Linien und ohne Bezeichnungstext ist möglich:

\inputListing{helloWorld.java}{java}

Jetzt lautet das Kommando:

\setupVerbatim{tex}
\begin{verbatim}
\inputListing{filename}{syntaxdef}
\end{verbatim}


\subsection{Zitieren und Referenzieren}
\label{subsec:zitieren}
Eine Quelle kann man mit dem Befehl \textbackslash cite\{...\}
zitieren. Dabei entsteht eine Ausgabe der Form \cite{han99a}.

Alle Literaturverweise müssen in einer Bib-TeX Datenbank gespeichert
sein. Zum einen steht am Lehrstuhl eine entsprechende Datenbank zur
Verfügung, zum anderen können Bib-TeX Einträge in entsprechend
formatierte Dateien im Verzeichnis .../Dokumentation/Latex/Bib/
gemacht werden. Zu beachten ist, dass je nach Wahl der Datenbank die
entsprechenden Zeilen in der Datei .../Dokumentation/Latex/main.tex
auskommentiert werden müssen.

In das Literaturverzeichnis werden nur im Dokument zitierte Quellen
auf\-ge\-nom\-men. Um eine Quelle, die nicht im Text zitiert wurde, mit
auf\-zu\-neh\-men, muss diese durch einen \texttt{\textbackslash nocite}
Eintrag in der Datei .../Dokumentation/Latex/main.tex angegeben
werden. Ein Beispiel ist in der enthaltenen Main-Datei zu finden.

Um in einem TeX-Dokument auf zuvor definierte Label zu verweisen,
kommt das Kommando \textbackslash ref zum Einsatz. Um ein Label zu
setzen wird der Befehl \textbackslash label verwendet. So ist es
möglich auf z.B. eine Abbildung wie Abbildung \ref{fig:demoGif} oder
auch auf einen Abschnitt des Dokuments wie dieses (Kapitel
\ref{subsec:zitieren}) zu verweisen.


\subsection{Weiterführende Literatur und HowTos}
\label{subsec:howto}
Im Internet sind eine Menge HowTos und Einstiegshilfen zu \LaTeX \index{\LaTeX}
zu finden. Außerdem ist eine Menge brauchbarer Bücher auf dem Markt
verfügbar. Wir möchten an dieser Stelle auf \cite{kop02} \cite{gms00}
verweisen. Eine Vielzahl an Büchern ist auch in den Bibliotheken der
Universität zu finden.


%%% Local Variables: 
%%% mode: latex
%%% TeX-master: t
%%% End: 

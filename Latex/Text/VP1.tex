\chapter{Vorarbeit}
\begin{itemize}
	\item Update auf Weezy
	\item symlink \verb#09_linux_xen# in /etc/grub.d erstellt
	\item \verb#/etc/network/interfaces#: \verb#auto eth0# added
	\item rootpw changed \verb#123#
	\item Nameserver in \verb#/etc/resolv.conf# auskommentiert da er grad ned antwortet (Routingproblem und andererseits antwortet er nicht auf Welt) 
\end{itemize}

\chapter{Storage}
\section{RAID}
Erstelle RAID über \verb#sdb2# und \verb#sdc1#
\setupVerbatim{bash}
\begin{verbatim}
apt-get install gdisk
gdisk /dev/sdb
# gdisk GPT Partition ~1M Rest Maximal erstellt.
sgdisk -R=/dev/sdc /dev/sdb
sgdisk -G /dev/sdb
# Partitionstabelle kopiert und GUID neu
mdadm --create /dev/md1 -n 2 -l 1 /dev/sdb2 /dev/sdc2
#	-n	Number of Devices
#	-l	RAID Level
dmesg
\end{verbatim}
\setupVerbatimOut
\begin{verbatim}
[  253.106706]  sdb: sdb1 sdb2
[  364.210743]  sdc: sdc1 sdc2
[  373.659167]  sdb: sdb1 sdb2
[  701.414204] md: bind<sdb2>
[  701.415194] md: bind<sdc2>
[  701.425611] md: raid1 personality registered for level 1
[  701.425853] bio: create slab <bio-1> at 1
[  701.425897] md/raid1:md1: not clean -- starting background reconstruction
[  701.425898] md/raid1:md1: active with 2 out of 2 mirrors
[  701.425913] md1: detected capacity change from 0 to 499971325952
[  701.433300]  md1: unknown partition table
[ 1223.952091] md: resync of RAID array md1
[ 1223.952093] md: minimum _guaranteed_  speed: 1000 KB/sec/disk.
[ 1223.952095] md: using maximum available idle IO bandwidth (but not more than 200000 KB/sec) for resync.
[ 1223.952097] md: using 128k window, over a total of 488253248k.
\end{verbatim}

\section{LVM}
\todo{Schönen text}
\setupVerbatim{bash}
\begin{verbatim}
#physical volume create
pvcreate /dev/md1 
#volumegroup create 
vgcreate storage /dev/md1
#2 Logical Volumes fuer VMs
lvcreate -n guest1 -L 10G storage
lvcreate -n guest2 -L 20G storage
\end{verbatim} 

\chapter{Open vSwitch}
\todo{Schönen text}
\setupVerbatim{bash}
\begin{verbatim}
apt-get instal openvswitch-brcompat openvswitch-switch openvswitch-datapath-dkms
modprobe openvswitch-mod
dmesg
\end{verbatim} 
\setupVerbatimOut
\begin{verbatim}
[ 2377.104677] openvswitch_mod: Open vSwitch switching datapath 1.4.2, built Apr 30 2013 15:47:50
\end{verbatim}
In Datei \verb#/etc/default/openvswitch-switch#: \verb#BRCOMPAT=yes# gesetzt
\setupVerbatimBash
\begin{verbatim}
/etc/init.d/openvswitch-switch start
ovs-vsctl add-br br-guest
ovs-vsctl show
\end{verbatim}
\setupVerbatimOut
\begin{verbatim}
b01c6804-1ad5-4294-98d1-b6aa0a8f6155
    Bridge br-guest
        Port br-guest
            Interface br-guest
                type: internal
    ovs_version: "1.4.2"
\end{verbatim} 
\setupVerbatimBash
\begin{verbatim}
ovs-vsctl add-port br-guest eth0
ovs-vsctl show
\end{verbatim}
\setupVerbatimOut
\begin{verbatim}
b01c6804-1ad5-4294-98d1-b6aa0a8f6155
     Bridge br-guest
        Port "eth0"
            Interface "eth0"
        Port br-guest
            Interface br-guest
                type: internal
    ovs_version: "1.4.2"
\end{verbatim}
\begin{itemize}
\item In \verb#/etc/network/interfaces#: \verb#eth0# durch \verb#br-guest# ersetzt. 
\item IP auf eth0 zurückgesetzt. 
\item Mittels Ping die Bridge getestet.
\end{itemize}
\chapter{VM}
\todo{Schönen text}
\setupVerbatim{bash}
\begin{verbatim}
xen-create-image --ip 192.168.10.12  --lvm=storage --hostname=guest1 --vcpus=2 --dist wheezy
\end{verbatim}
\setupVerbatimOut
\begin{verbatim}                                                  
WARNING                                           
-------                                           
                                                  
  You appear to have a missing vif-script, or network-script, in the
 Xen configuration file /etc/xen/xend-config.sxp. 
                                                  
  Please fix this and restart Xend, or your guests will not be able
 to use any networking!                           
                                                  
WARNING:  No gateway address specified!           
WARNING:  No netmask address specified!           
                                                  
General Information                               
--------------------                              
Hostname       :  guest1                          
Distribution   :  wheezy                          
Mirror         :  http://cdn.debian.net/debian/   
Partitions     :  swap            128Mb (swap)    
                  /               4Gb   (ext3)    
Image type     :  full                            
Memory size    :  128Mb                           
Kernel path    :  /boot/vmlinuz-3.2.0-4-amd64     
Initrd path    :  /boot/initrd.img-3.2.0-4-amd64  
                                                  
Networking Information                            
----------------------                            
IP Address 1   : 192.168.10.12 [MAC: 00:16:3E:95:61:15]
                                                  
                                                  
Creating swap on /dev/storage/guest1-swap         
Done                                              
                                                  
Creating ext3 filesystem on /dev/storage/guest1-disk
Done                                              
Installation method: debootstrap
Done

Running hooks
Done

No role scripts were specified.  Skipping

Creating Xen configuration file
Done

No role scripts were specified.  Skipping
Setting up root password
Generating a password for the new guest.
All done


Logfile produced at:
         /var/log/xen-tools/guest1.log

Installation Summary
---------------------
Hostname        :  guest1
Distribution    :  wheezy
IP-Address(es)  :  192.168.10.12 
RSA Fingerprint :  01:06:31:35:f0:d4:f0:70:54:d3:0f:f2:d4:90:ba:e1
Root Password   :  JzXi4Ufg

\end{verbatim}
Passwort auf 123 geändert!

\begin{itemize}
	\item Memory in \verb#/etc/xen/guest1.cfg# erhöht auf 512 MB
	\item VM mit \verb#xm create /etc/xen/guest1.cfg# gestartet
	\item \verb#vif1.0# wurde erstellt und mit br-guest verbunden
	\item mittels \verb#xm console guest1# in VM gewechselt.
	\item VM hat Konnectivität. Getestet mittels \verb#ping#.
	\item \verb#Ctrl+AltGr+]# beendet \verb#xm console#
\end{itemize}

\section{2. VM sowie Zusätzliche Bridge für VM-Interconnect}
\begin{itemize}
	\item \verb#ovs-vsctl add-br br-vmonly#
	\item \verb#xen-create-image --ip 192.168.10.13  --lvm=storage --hostname=guest2 --vcpus=2 --bridge=br-guest#
	\item Config von Guest1 angepasst um 2. Intervace auf vmonly zu erhalten. Shutdown - Create
	\item Guest2 gestartet
	\item Host lernt Routing und NAT
	\item iperf auf guest1 zu guest2 \verb#iperf -c 192.168.2.2#
\end{itemize}

\setupVerbatimOut
\begin{verbatim}
------------------------------------------------------------
Client connecting to 192.168.2.2, TCP port 5001
TCP window size: 23.5 KByte (default)
------------------------------------------------------------
[  3] local 192.168.2.1 port 57804 connected with 192.168.2.2 port 5001
[ ID] Interval       Transfer     Bandwidth
[  3]  0.0-10.0 sec  9.21 GBytes  7.91 Gbits/sec


------------------------------------------------------------
Client connecting to 192.168.1.2, TCP port 5001
TCP window size: 23.5 KByte (default)
------------------------------------------------------------
[  3] local 192.168.1.1 port 56132 connected with 192.168.1.2 port 5001
[ ID] Interval       Transfer     Bandwidth
[  3]  0.0-10.0 sec  10.6 GBytes  9.12 Gbits/sec
\end{verbatim}
tcpdump zeigt interessante Ethernet-Framegrößen 65226. Ping mit höherer MTU funktioniert wenn man Interfaces in Gästen, Bridge unf vifs entsprechend anpasst.
\setupVerbatimOut
\begin{verbatim}
17:45:48.414535 00:16:3e:32:d0:3e > 00:16:3e:95:de:ad, ethertype IPv4 (0x0800), length 65226: (tos 0x0, ttl 64, id 25535, offset 0, flags [DF], proto TCP (6), length 65212)
    192.168.1.2.57025 > 192.168.1.1.5001: Flags [.], seq 14811337:14876497, ack 0, win 913, options [nop,nop,TS val 396548 ecr 356977], length 65160
17:45:48.414540 00:16:3e:32:d0:3e > 00:16:3e:95:de:ad, ethertype IPv4 (0x0800), length 65226: (tos 0x0, ttl 64, id 25580, offset 0, flags [DF], proto TCP (6), length 65212)
    192.168.1.2.57025 > 192.168.1.1.5001: Flags [P.], seq 14876497:14941657, ack 0, win 913, options [nop,nop,TS val 396548 ecr 356977], length 65160
17:45:48.414542 00:16:3e:32:d0:3e > 00:16:3e:95:de:ad, ethertype IPv4 (0x0800), length 65226: (tos 0x0, ttl 64, id 25625, offset 0, flags [DF], proto TCP (6), length 65212)
    192.168.1.2.57025 > 192.168.1.1.5001: Flags [.], seq 14941657:15006817, ack 0, win 913, options [nop,nop,TS val 396548 ecr 356977], length 65160
17:45:48.414559 00:16:3e:95:de:ad > 00:16:3e:32:d0:3e, ethertype IPv4 (0x0800), length 66: (tos 0x0, ttl 64, id 54265, offset 0, flags [DF], proto TCP (6), length 52)
    192.168.1.1.5001 > 192.168.1.2.57025: Flags [.], cksum 0x837a (incorrect -> 0x86f4), seq 0, ack 14876497, win 16397, options [nop,nop,TS val 356977 ecr 396548], length 0
17:45:48.414571 00:16:3e:95:de:ad > 00:16:3e:32:d0:3e, ethertype IPv4 (0x0800), length 66: (tos 0x0, ttl 64, id 54266, offset 0, flags [DF], proto TCP (6), length 52)
    192.168.1.1.5001 > 192.168.1.2.57025: Flags [.], cksum 0x837a (incorrect -> 0x886b), seq 0, ack 14941657, win 16397, options [nop,nop,TS val 356977 ecr 396548], length 0
17:45:48.414576 00:16:3e:95:de:ad > 00:16:3e:32:d0:3e, ethertype IPv4 (0x0800), length 66: (tos 0x0, ttl 64, id 54267, offset 0, flags [DF], proto TCP (6), length 52)^C
    192.168.1.1.5001 > 192.168.1.2.57025: Flags [.], cksum 0x837a (incorrect -> 0x91f8), seq 0, ack 15006817, win 14327, options [nop,nop,TS val 356977 ecr 396548], length 0
\end{verbatim}






